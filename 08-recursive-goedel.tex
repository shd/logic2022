\documentclass[aspectratio=169]{beamer}
\usepackage[utf8]{inputenc}
\usepackage[english,russian]{babel}
\usepackage{cancel}
\usepackage{amssymb}
\usepackage{stmaryrd}
\usepackage{cmll}
\usepackage{graphicx}
\usepackage{amsthm}
\usepackage{tikz}
\usepackage{multicol}
\usetikzlibrary{patterns}
\usepackage{chronosys}
\usepackage{proof}
\usepackage{multirow}
\setbeamertemplate{navigation symbols}{}
%\usetheme{Warsaw}

\newtheorem{thm}{Теорема}[section]
\newtheorem{dfn}{Определение}[section]
\newtheorem{lmm}{Лемма}[section]
\newtheorem{exm}{Пример}[section]
\newtheorem{snote}{Пояснение}[section]

\newcommand{\divisible}%
{\mathrel{\lower.2ex%
\vbox{\baselineskip=0.7ex\lineskiplimit=0pt%
\kern6pt \hbox{.}\hbox{.}\hbox{.}}%
}}

\begin{document}

\newcommand\doubleplus{+\kern-1.3ex+\kern0.8ex}
\newcommand\mdoubleplus{\ensuremath{\mathbin{+\mkern-10mu+}}}

\begin{frame}{Арифметизация логики}
\end{frame}

\begin{frame}{Общие замечания}
\begin{itemize}
\item Рассматриваем функции $\mathbb{N}^n_0\to\mathbb{N}_0$.
\item Обозначим вектор $\langle x_1, x_2, \dots, x_n\rangle$ как $\overrightarrow{x}$.
\end{itemize}
\end{frame}

\begin{frame}{Примитивно-рекурсивные функции}

\begin{dfn}[Примитивы Z, N, U, S]
\begin{enumerate}
\item Примитив <<Ноль>> ($Z$) \vspace{-0.3cm}
$$Z: \mathbb{N}_0\to\mathbb{N}_0,\ \ \ \ \ Z(x_1) = 0$$\vspace{-0.5cm}\pause
\item Примитив <<Инкремент>> ($N$) \vspace{-0.3cm}
$$N: \mathbb{N}_0\to\mathbb{N}_0,\ \ \ \ \ N(x_1) = x_1+1$$\vspace{-0.5cm}\pause
\item Примитив <<Проекция>> ($U$) — семейство функций; пусть $k,n \in \mathbb{N}_0, k \le n$\vspace{-0.3cm}
$$U^k_n: \mathbb{N}^n_0 \to \mathbb{N}_0,\ \ \ \ \ U^k_n(\overrightarrow{x}) = x_k$$\vspace{-0.5cm}\pause
\item Примитив <<Подстановка>> ($S$) --- семейство функций; пусть $g: \mathbb{N}^k_0 \to \mathbb{N}_0,\ \ f_1,\dots,f_k: \mathbb{N}^n_0 \to \mathbb{N}_0$
 \vspace{-0.3cm}
$$S\langle g,f_1,f_2,\dots,f_k \rangle (\overrightarrow{x}) = g(f_1(\overrightarrow{x}),\dots,f_k(\overrightarrow{x}))$$
\end{enumerate}
\end{dfn}
\end{frame}

\begin{frame}[fragile]{Примитивная рекурсия}
\begin{dfn}[примитив <<примитивная рекурсия>>, $R$]
Пусть $f: \mathbb{N}^n_0\to\mathbb{N}_0$ и $g: \mathbb{N}^{n+2}_0 \to\mathbb{N}_0$.
Тогда $R\langle f,g\rangle: \mathbb{N}^{n+1}_0\to\mathbb{N}_0$, причём

$$R\langle f,g\rangle(\overrightarrow{x},y)=
 \left\{\begin{array}{ll} 
  f(\overrightarrow{x}), &y=0\\
  g(\overrightarrow{x},y-1,R\langle f,g\rangle (\overrightarrow{x},y-1)), &y > 0
\end{array}\right.$$
\end{dfn}\pause

\vspace{-0.3cm}
\begin{snote}\vspace{-0.5cm}
\begin{verbatim}
                    res := f(x1...xn);
                    for yi = 0 to y-1 do
                        res := g(x1...xn,yi,res);
\end{verbatim}
\end{snote}\pause

\vspace{-0.3cm}
\begin{exm}\vspace{-0.5cm}
$$\begin{array}{ll}
    R\langle f,g\rangle(\overrightarrow{x},3) &= g(\overrightarrow{x},2,R\langle f,g\rangle(\overrightarrow{x},2)) \\\pause
 &=   g(\overrightarrow{x},2,g(\overrightarrow{x},1,R\langle f,g\rangle(\overrightarrow{x},1))) \\\pause
 &=   g(\overrightarrow{x},2,g(\overrightarrow{x},1,g(\overrightarrow{x},0,R\langle f,g\rangle(\overrightarrow{x},1)))) \\\pause
 &=  g(\overrightarrow{x},2,g(\overrightarrow{x},1,g(\overrightarrow{x},0,f(\overrightarrow{x}))))
\end{array}$$
\end{exm}

%$$R\langle f,g\rangle(\overrightarrow{x},y) = \texttt{foldl}\ (\lambda r.\lambda t.g(\overrightarrow{x},t,r))\ f(\overrightarrow{x})\ [0..y-1]$$

\end{frame}

\begin{frame}{Примитивно-рекурсивные функции}
\begin{dfn}Функция $f$ --- примитивно-рекурсивна, если может быть
выражена как композиция примитивов $Z$, $N$, $U$, $S$ и $R$.
\end{dfn}\pause

\begin{thm}
$f(x) = x+2$ примитивно-рекурсивна
\end{thm}\pause

\begin{proof}
$f = S \langle N,N \rangle$
\vspace{0.5cm}\pause

{\color{gray}
$N(x)=x+1$

$S\langle g,f\rangle (x) = g(f(x))$
}\pause

\vspace{0.5cm}
$f,g = N$\pause

$S\langle N,N\rangle (x) = N(N(x)) = (x+1)+1$
\end{proof}
\end{frame}

\begin{frame}{Примитивно-рекурсивные функции: $x+y$}
\begin{lmm}
$f(a,b) = a+b$ примитивно-рекурсивна
\end{lmm}
\begin{proof} $f = R \langle U^1_1, S\langle N, U^3_3\rangle \rangle$:
\vspace{0.2cm}\pause

{\color{gray}
$R\langle f,g\rangle(x,y)
=\left\{\begin{array}{ll} 
  f(x), &y=0\\
  g(x,y-1,R\langle f,g\rangle (x,y-1)), &y > 0
\end{array}\right.$ 

%$U^1_1(x) = x$

%$U^3_3(x,y,r) = r$
}\vspace{0.2cm}\pause

\begin{itemize}
\item База. $R\langle U^1_1, S\langle N, U^3_3\rangle \rangle(x,0) = U^1_1(x) = x$\pause

\item Переход. $R\langle U^1_1, S\langle N, U^3_3\rangle \rangle(x,y+1) = $\pause

$... = S\langle N, U^3_3\rangle (x,y,R\langle U^1_1(x), S\langle N, U^3_3\rangle\rangle (x,y) ) =$\pause

$... = S\langle N, U^3_3\rangle (x,y,x + y) = $\pause 

$... = N(x+y) = x+y+1$
\end{itemize}

\end{proof}
\end{frame}

\begin{frame}[fragile]{Какие функции примитивно-рекурсивные?}
\begin{enumerate}
\item Сложение, вычитание\pause
\item Умножение, деление\pause
\item Вычисление простых чисел\pause
\item Неформально: все функции, вычисляемые конечным числом вложенных циклов \verb!for!:

\begin{verbatim}
for (int i1 = 0; i_1 < g1(x1...xn); i1++) {
    for (int i2 = 0; i_2 < g2(x1...xn,i1); i2++) {
        ...
           for (int ik = 0; ik < gk(x1...xn,i1,i2...); ik++) {
               // выражение без циклов
           }
        ...
    }
}
\end{verbatim}
\end{enumerate}
\end{frame}

\begin{frame}[fragile]{Общерекурсивные функции}
\begin{dfn}
Функция --- общерекурсивная, если может быть построена при помощи
примитивов $Z$, $N$, $U$, $S$, $R$ и примитива минимизации:
$$M\langle f \rangle (x_1,x_2,\dots,x_n) = \min\{y: f(x_1,x_2,\dots,x_n,y) = 0\}$$
Если $f(x_1,x_2,\dots,x_n,y) > 0$ при любом $y$, результат неопределён.
\end{dfn}\pause

%Это --- цикл \texttt{while}: 

Пример:

Пусть $f(x,y) = x-y^2$, тогда $\lceil\sqrt{x}\rceil = M\langle f\rangle (x)$

\begin{verbatim}
int sqrt(int x) {
    int y = 0;
    while (x-y*y > 0) y++;
    return y;
}
\end{verbatim}

\end{frame}

\begin{frame}{Выразительная сила}
\begin{dfn}
Функция Аккермана:
$$A(m,n) = \left\{\begin{array}{ll}
  n+1,&m = 0\\
  A(m-1,1),&m > 0, n = 0\\
  A(m-1,A(m,n-1)),&m > 0, n > 0
\end{array}\right.$$
\end{dfn}

\begin{thm}
Функция Аккермана --- общерекурсивная, но не примитивно-рекурсивная.
\end{thm}\pause

\begin{dfn}
Тезис Чёрча для общерекурсивных функций: любая эффективно-вычислимая
функция $\mathbb{N}^k_0\to\mathbb{N}_0$ является общерекурсивной.
\end{dfn}
\end{frame}

\begin{frame}{Новые обозначения}
\begin{dfn}
Запись вида $\psi(\theta_1,\dots,\theta_n)$ означает
$\psi[x_1:=\theta_1,\dots,x_n:=\theta_n]$
\end{dfn}\pause

\begin{dfn}[Литерал числа]
$$\overline{a} = \left\{\begin{array}{ll} 0, &\mbox{если } a = 0\\
                (\overline{b})', &\mbox{если } a = b+1
\end{array}\right.$$
\end{dfn}\pause

Пример: пусть $\psi := x_1 = 0$. \pause Тогда $\psi(\overline{3})$ соответствует формуле $0''' = 0$

\end{frame}

\begin{frame}{Выразимость отношений в Ф.А.}
\begin{dfn}
Будем говорить, что отношение $R\subseteq \mathbb{N}^n_0$ выразимо в ФА, 
если существует формула $\rho$, что:
\begin{enumerate}
\item если $\langle a_1,\dots,a_n \rangle \in R$, то $\vdash \rho(\overline{a_1},\dots,\overline{a_n})$
\item если $\langle a_1,\dots,a_n \rangle \notin R$, то $\vdash \neg\rho(\overline{a_1},\dots,\overline{a_n})$
\end{enumerate}
\end{dfn}\pause

\begin{thm} отношение <<равно>> выразимо в Ф.А.:
$R = \{ \langle x,x \rangle\ |\ x \in \mathbb{N}_0 \}$
\end{thm}\pause

\begin{proof}
Пусть $\rho := x_1=x_2$. \pause Тогда:
\begin{itemize}
\item $\vdash p = p$ при $p := \overline{k}$ при всех $k \in \mathbb{N}_0$: \pause $\vdash 0=0$, $\vdash 0'=0'$, $\vdash 0''=0''$, ... \pause
\item $\vdash \neg p = q$ при $p := \overline{k}$, $q := \overline{s}$ при всех $k,s \in \mathbb{N}_0$ и $k \ne s$.

\pause $\vdash \neg 0 = 0'$, $\vdash \neg 0 = 0''$, $\vdash \neg 0''' = 0'$, ...
\end{itemize}
\end{proof}
\end{frame}

\begin{frame}{Представимость функций в Ф.А.}
\begin{dfn}
Будем говорить, что функция $f: \mathbb{N}^n_0\to\mathbb{N}_0$ представима в ФА, 
если существует формула $\varphi$, что:
\begin{enumerate}
\item если $f(a_1,\dots,a_n) = u$, то $\vdash \varphi(\overline{a_1},\dots,\overline{a_n},\overline{u})$
\item если $f(a_1,\dots,a_n) \ne u$, то $\vdash \neg\varphi(\overline{a_1},\dots,\overline{a_n},\overline{u})$
\item для всех $a_i \in \mathbb{N}_0$ выполнено $\vdash (\exists x.\varphi(\overline{a_1},\dots,\overline{a_n},x)) \with 
   (\forall p.\forall q.\varphi(\overline{a_1},\dots,\overline{a_n},p)\with \varphi(\overline{a_1},\dots,\overline{a_n},q)\rightarrow p=q)$
\end{enumerate}
\end{dfn}
\end{frame}

\begin{frame}{Соответствие рекурсивных и представимых функций}
\begin{thm}Любая рекурсивная функция представима в Ф.А.\end{thm}\pause

\begin{thm}Любая представимая в Ф.А. функция рекурсивна.\end{thm}
\end{frame}

\begin{frame}{Примитивы Z, N, U представимы в Ф.А.}
\begin{thm}
Примитивы $Z$, $N$ и $U^k_n$ представимы в Ф.А.
\end{thm}\pause

\begin{proof}
\begin{itemize}
\item $\zeta(x_1,x_2) := x_2=0$, \pause формальнее: $\zeta(x_1,x_2) := x_1=x_1 \with x_2=0$\pause
\item $\nu(x_1,x_2) := x_2=x_1'$\pause
\item $\upsilon(x_1,\dots,x_n,x_{n+1}) := x_k = x_{n+1}$ \pause \vspace{0.1cm}

формальнее: 
     $\upsilon(x_1,\dots,x_n,x_{n+1}) := (\underset{i\ne k,n+1}{\with} x_i=x_i) \with x_k = x_{n+1}$
\end{itemize}
\end{proof}
\end{frame}

\begin{frame}{Примитив S представим в Ф.А.}

$$S\langle f,g_1,\dots,g_k\rangle(x_1,\dots,x_n) = f(g_1(x_1,\dots,x_n),\dots,g_k(x_1,\dots,x_n))$$\pause
\begin{thm}
Пусть функции $f,g_1,\dots,g_k$ представимы в Ф.А. Тогда $S\langle f,g_1,\dots,g_k \rangle$ представима в Ф.А.
\end{thm}\pause

\begin{proof}
Пусть $f$, $g_1$, ..., $g_k$ представляются формулами $\varphi$, $\gamma_1$, ..., $\gamma_k$. \pause

Тогда 
$S\langle f,g_1,\dots,g_k\rangle$ будет представлена формулой
$$\exists g_1.\dots.\exists g_k.\varphi(g_1,\dots,g_k,x_{n+1})\with\gamma_1(x_1,\dots,x_n,g_1)\with\dots\with\gamma_k(x_1,\dots,x_n,g_k)$$
\end{proof}

\end{frame}

\begin{frame}{$\beta$-функция Гёделя}
Задача: закодировать последовательность натуральных чисел произвольной длины.\pause
\begin{dfn}$\beta$-функция Гёделя: $\mathcal{\beta}(b,c,i) := b \% (1 + (i+1) \cdot c)$\\
Здесь (\%) --- остаток от деления.
\end{dfn}\pause
\begin{thm}$\beta$-функция Гёделя представима в Ф.А. формулой
$$\hat{\beta}(b,c,i,d) := \exists q.(b = q \cdot (1 + c \cdot (i+1)) + d) \& (d < 1 + c \cdot (i+1))$$
\end{thm}\pause\vspace{-0.3cm}
Деление $b$ на $x$ с остатком: найдутся частное $(q)$ и остаток $(d)$, что
$b = q\cdot x + d$ и $0 \le d < x$.
\pause
\begin{thm}Если $a_0, \dots, a_n \in \mathbb{N}_0$, то найдутся такие $b,c \in \mathbb{N}_0$, что
$a_i = \beta(b,c,i)$
\end{thm}
\end{frame}

\begin{frame}{Доказательство свойства $\beta$-функции}
\begin{thm}Китайская теорема об остатках (вариант формулировки): если $u_0, \dots, u_n$ --- попарно 
взаимно-просты, и $0 \le a_i < u_i$, то существует такой $b$, что $a_i = b \% u_i$.
\end{thm}\pause
Положим $c = \max(a_0,\dots,a_n,n)!$ и $u_i = 1+c\cdot(i+1)$.\pause
\begin{itemize}
\item $\text{НОД}(u_i,u_j) = 1$, если $i \ne j$.\pause

Пусть $p$ --- простое, $u_i\divisible p$ и $u_j\divisible p$ ($i < j$). \pause
Заметим, что $u_j-u_i = c \cdot (j-i)$. Значит, $c\divisible p$ или $(j-i)\divisible p$. \pause
Так как $j-i \le n$, то $c \divisible (j-i)$, потому если и $(j-i)\divisible p$, всё равно $c \divisible p$. \pause
Но и $(1+c\cdot(i+1))\divisible p$, отсюда $1 \divisible p$ --- что невозможно. \pause
\item $0 \le a_i < u_i$.\pause
\end{itemize}
Условия китайской теоремы об остатках выполнены и найдётся $b$, что \vspace{-0.2cm}
$$a_i = b \% (1 + c\cdot(i+1)) = \beta(b,c,i)$$
\qed
\end{frame}

\begin{frame}{Примитив <<примитивная рекурсия>> представим в Ф.А.}

Пусть $f: \mathbb{N}^n_0\to\mathbb{N}_0$ и $g:\mathbb{N}^{n+2}_0\to\mathbb{N}_0$ представлены формулами $\varphi$ и $\gamma$. 

Зафиксируем $x_1, \dots, x_n, y \in \mathbb{N}_0$.\pause\vspace{0.3cm}

\begin{tabular}{lll}
Шаг вычисления & Об. & Утверждение в Ф.А.\\
$R\langle f,g\rangle (x_1,\dots,x_n,0) = f(x_1,\dots,x_n)$ & $a_0$ & $\vdash \varphi(\overline{x_1},\dots,\overline{x_n},\overline{a_0})$\\ \pause
$R\langle f,g\rangle (x_1,\dots,x_n,1) = g(x_1,\dots,x_n,0,a_0)$ & $a_1$ & $\vdash \gamma(\overline{x_1},\dots,\overline{x_n},0,\overline{a_1})$\\ \pause
$\dots$\\
$R\langle f,g\rangle (x_1,\dots,x_n,y) = g(x_1,\dots,x_n,y-1,a_{y-1})$ & $a_y$ & $\vdash \gamma(\overline{x_1},\dots,\overline{x_n},\overline{y-1},\overline{a_y})$
\end{tabular}\pause\vspace{0.3cm}

По свойству $\beta$-функции, найдутся $b$ и $c$, что
$\beta (b,c,i) = a_i$ для $0 \le i \le y$.\pause

\begin{thm}Примитив $R\langle f,g\rangle$ представим в Ф.А. формулой $\rho(x_1,\dots,x_n,y,a)$:\vspace{-0.1cm}
$$\begin{array}{l}\exists b. \exists c. (\exists a_0. \hat{\beta} (b,c,0,a_0) \& \varphi (x_1,...x_n, a_0)) \\
       \&\;\;\;\;\forall k.k < y \rightarrow \exists d . \exists e . \hat{\beta} (b,c,k,d) \& \hat{\beta} (b,c,k',e) \& \gamma (x_1,..x_n,k,d,e) \\
       \&\;\;\;\;\hat{\beta} (b,c,y,a) 
\end{array}$$
\end{thm}
\end{frame}

\begin{frame}{Представимость рекурсивных функций в Ф.А.}
\begin{thm}Пусть функция $f:\mathbb{N}^{n+1}_0 \to \mathbb{N}_0$ представима в Ф.А.
формулой $\varphi(x_1,\dots,x_{n},y,r)$. Тогда примитив $M\langle f\rangle$ представим в Ф.А. 
формулой $$\mu(x_1,\dots,x_n,y) := \neg\varphi(x_1,\dots,x_n,y,0) \with \forall u.u < y \to \varphi(x_1,\dots,x_n,u,0)$$
\end{thm}\pause

\begin{thm}Если $f$ --- рекурсивная функция, то она представима
в Ф.А.
\end{thm}
\begin{proof}Индукция по структуре $f$.
\end{proof}
\end{frame}

\begin{frame}{Рекурсивность представимых функций в Ф.А.}
Фиксируем $f$ и $x_1, x_2, \dots, x_n$. Обозначим $y = f(x_1,x_2,\dots,x_n)$. \pause
По представимости нам известна $\varphi$, что $\vdash \varphi(\overline{x_1},\overline{x_2},\dots,\overline{x_n},\overline{y})$. \pause
Давайте просто переберём все результаты и доказательства!\pause

\begin{enumerate}
\item Закодируем доказательства натуральными числами.\pause
\item Напишем рекурсивную функцию, проверяющую доказательства на корректность.\pause
\item Параллельный перебор значений и доказательств: $s = 2^y \cdot 3^p$. Переберём все $s$, по $s$ получим $y$ и $p$.
Проверим, что $p$ --- код доказательства $\vdash \varphi(\overline{x_1},\overline{x_2},\dots,\overline{x_n},\overline{y})$.
\end{enumerate}
\end{frame}

\begin{frame}{Гёделева нумерация}

\begin{enumerate}
\item Отдельный символ.\vspace{0.2cm}

\begin{tabular}{lc|lc||cll}
Номер & Символ & Номер & Символ & Имя & $k,n$ & Гёделев номер\\
3 & ( &               17 & $\&$ &  0 & $0,0$ & $27 + 6$\\
5 & ) &               19 & $\forall$ & $(')$ & $0,1$ & $27 + 6 \cdot 3$\\
7 & , &               21 & $\exists$ & $(+)$ & $0,2$ & $27 + 6 \cdot 9$\\
9 & . &               23 & $\vdash$ & $(\cdot)$ & $1,2$ & $27 + 6 \cdot 2 \cdot 9$\\
11 & $\neg$ &         $25 + 6\cdot k$ & $x_k$ & $(=)$ & $0,2$ & $29 + 6 \cdot 9$\\
13 & $\rightarrow$ &  $27 + 6\cdot 2^k \cdot 3^n$ & $f_k^n$\\
15 & $\vee$ &         $29 + 6\cdot 2^k \cdot 3^n$ & $P_k^n$ 
\end{tabular}\pause

\item Формула. $\phi \equiv s_0s_1\dots s_{n-1}$. Гёделев номер: $\ulcorner\phi\urcorner = 2^{\ulcorner s_0\urcorner}\cdot 3^{\ulcorner s_1\urcorner} 
\cdot \dots \cdot p_{n-1}^{\ulcorner s_{n-1}\urcorner}$.\pause

\item Доказательство. $\Pi = \delta_0\delta_1\dots\delta_{k-1}$, его гёделев номер: $\ulcorner\Pi\urcorner =
2^{\ulcorner \delta_0\urcorner}\cdot 3^{\ulcorner \delta_1\urcorner} \cdot \dots \cdot p_{k-1}^{\ulcorner \delta_{k-1}\urcorner}$
\end{enumerate}
\end{frame}

\begin{frame}{Проверка доказательства на корректность}
\begin{thm}Следующая функция рекурсивна:
$$\text{proof}(f,x_1,x_2,\dots,x_n,y,p) = \left\{\begin{array}{ll} 
1, & \mbox{если} \vdash\phi(\overline{x_1},\overline{x_2},\dots,\overline{x_n},\overline{y}),\\
   & p \mbox{ --- гёделев номер вывода}, f = \ulcorner\phi\urcorner \\ 
0, & \mbox{иначе}
\end{array}\right.$$\vspace{-0.1cm}\pause
\end{thm}
\begin{proof}[Идея доказательства]
\begin{enumerate}
%\item По представимости, $\vdash\phi(\overline{x_1},\overline{x_2},\dots,\overline{x_n},\overline{y})$ 
%эквивалентно $f(x_1,x_2,\dots,x_n)=y$.\pause
\item Проверка доказательства вычислима.\pause
\item Согласно тезису Чёрча, любая вычислимая функция
вычислима с помощью рекурсивных функций.
\end{enumerate}
\end{proof}
\end{frame}

\begin{frame}{Перебор доказательств}
\begin{lmm}Следующие функции рекурсивны:
\begin{enumerate}
\item Функции $\text{plog}_k(n) = \max\{p: n \divisible k^p\}$, $\text{fst}(x) = \text{plog}_2(x)$ и $\text{snd}(x) = \text{plog}_3(x)$.\pause
\item Числовые литералы: $\overline{k}: \mathbb{N}_0 \rightarrow \mathbb{N}_0$, $\overline{k}(x) = k$.\pause
\end{enumerate}
\end{lmm}
\begin{thm}Если $f: \mathbb{N}^n_0\rightarrow\mathbb{N}_0$, и $f$ представима в Ф.А. формулой $\varphi$, то $f$ --- рекурсивна.
\end{thm}\pause
\begin{proof}
Пусть заданы $x_1,x_2,\dots,x_n$. Ищем $\langle y, p\rangle$, 
что $\text{proof}(\ulcorner\varphi\urcorner,x_1,x_2,\dots,x_n,y,p)=1$, \pause
напомним: $y = f(x_1,x_2,\dots,x_n)$, $p = \ulcorner\Pi\urcorner$,
$\Pi$ --- доказательство $\varphi(\overline{x_1},\overline{x_2},\dots,\overline{x_n},\overline{y})$. \pause
$$f = S \langle \text{fst}, M\langle S \langle \text{proof}, \overline{\ulcorner\varphi\urcorner}, U^1_{n+1}, U^2_{n+1}, \dots, U^n_{n+1}, 
  S \langle \text{fst},U^{n+1}_{n+1}\rangle, S\langle \text{snd}, U^{n+1}_{n+1}\rangle \rangle \rangle \rangle$$
\end{proof}
\end{frame}

\begin{frame}{}
\begin{center}\LARGE Первая теорема Гёделя о неполноте арифметики\end{center}
\end{frame}

\begin{frame}{Парадокс лжеца}
\begin{center}Предложение, указанное в центре данного слайда --- ложное.\end{center}
\end{frame}

\begin{frame}{Проблема останова}
\begin{thm}Невозможно разработать программу (функцию) \texttt{bool p (string source, string arg)},
возвращающую \texttt{true}, если программа с исходным кодом \texttt{source} имеет один аргумент
типа \texttt{string} и оканчивает работу, если ей передать на вход значение \texttt{arg}. 
\end{thm}

\begin{proof}
\begin{itemize}
\item Определим программу \\\texttt{bool s (std::string arg) \{\\
\hspace{1cm}if (s (arg)) \{ while (true); \}\\
\hspace{1cm}return true;\\
\}}
\item Пусть её полный исходный код --- в переменной \texttt{source}. 
\item Что вернёт \texttt{p (source, source)}?
\end{itemize}
\end{proof}
\end{frame}

\begin{frame}{Самоприменимость}
\begin{dfn}Определим функцию $W_1$:
$W_1(x,p)=1$, если
$x = \ulcorner \xi \urcorner$, где $\xi$ --- формула с единственной свободной
переменной $x_1$, а $p$ --- доказательство самоприменения $\xi$:
$$\vdash \xi(\overline{\ulcorner \xi \urcorner})$$
$W_1(x,p)=0$, если это не так.
\end{dfn}
\end{frame}

\begin{frame}{Теорема Гёделя, вспомогательные определения}
\begin{thm}Существует формула $\omega_1$ со свободными переменными $x_1$ и $x_2$ такая, что:
\begin{enumerate}
\item $\vdash \omega_1(\overline{\ulcorner \varphi \urcorner},\overline{p})$, если $p$ --- гёделев номер
доказательства самоприменения $\varphi$;
\item $\vdash \neg\omega_1(\overline{\ulcorner \varphi \urcorner},\overline{p})$ иначе.
\end{enumerate}
\end{thm}
\begin{proof}Опираясь на рекурсивность функции proof, легко показать рекурсивность $W_1$.
Значит, эта функция представима в формальной арифметике некоторой формулой $\tau_1$.
Возьмём $\omega_1(x_1,x_2) := \tau_1(x_1,x_2,\overline{1})$.
\end{proof}

\begin{dfn}
Определим формулу $\sigma(x) := \forall p.\neg\omega(x,p)$
\end{dfn}
\end{frame}

\begin{frame}{Омега-непротиворечивость}
\begin{dfn}Если для любой формулы $\phi(x)$ из $\vdash\phi(0)$, $\vdash\phi(\overline{1})$,
$\vdash\phi(\overline{2})$, ... выполнено $\not\vdash\exists x.\neg\phi(x)$, 
то теория \emph{омега-непротиворечива}.
\end{dfn}
\begin{thm}Омега-непротиворечивость влечёт непротиворечивость\end{thm}
\begin{proof}Пусть $\phi(x) \equiv (x=x)\rightarrow(x=x)\rightarrow(x=x)$. Тогда $\vdash\phi(x)$ при всех $x$.
Тогда $\not\vdash\exists x.\neg\phi(x)$ --- то есть существует недоказуемая формула, т.е. теория непротиворечива.
\end{proof}
\end{frame}

\begin{frame}{Первая теорема Гёделя о неполноте арифметики}

\begin{thm}{Первая теорема Гёделя о неполноте арифметики}
\begin{itemize}
\item Если формальная арифметика непротиворечива, то $\not\vdash\sigma(\overline{\ulcorner\sigma\urcorner})$.
\item Если формальная арифметика $\omega$-непротиворечива, то $\not\vdash\neg\sigma(\overline{\ulcorner\sigma\urcorner})$.
\end{itemize}
\end{thm}
\end{frame}

\end{document}
